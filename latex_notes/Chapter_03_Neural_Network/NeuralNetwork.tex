\documentclass{article}
\usepackage[a4paper, top=2cm,right=1cm,left=1cm,bottom=0.5cm]{geometry}
\usepackage{url}

\usepackage{ifthen}
\usepackage{xcolor}
\usepackage{amsmath}
\usepackage{tikz} 

\usepackage{xcolor}
\usepackage{listings}
\lstset{
  language=Python,
  basicstyle=\ttfamily\footnotesize,
  keywordstyle=\color{blue},
  stringstyle=\color{red},
  commentstyle=\color{gray},
  showstringspaces=false,
  breaklines=true,
  numbers=left,            % ← Line numbers on the left
  numberstyle=\tiny\color{gray}, % Style of line numbers
  numbersep=2pt            % Space between numbers and code
}

\usepackage{multicol}
\usepackage[compact]{titlesec}

\usepackage{mathtools}
\usepackage{siunitx}
\sisetup{inter-unit-product=\ensuremath{{}\cdot{}}, exponent-product = \cdot}
\DeclareSIUnit\bar{bar}

\usepackage[version=4]{mhchem} 

\makeatletter
\newsavebox\myboxA
\newsavebox\myboxB
\newlength\mylenA
\newcommand*\overbar[2][0.75]{%
    \sbox{\myboxA}{$\m@th#2$}%
    \setbox\myboxB\null% Phantom box
    \ht\myboxB=\ht\myboxA%
    \dp\myboxB=\dp\myboxA%
    \wd\myboxB=#1\wd\myboxA% Scale phantom
    \sbox\myboxB{$\m@th\overline{\copy\myboxB}$}%  Overlined phantom
    \setlength\mylenA{\the\wd\myboxA}%   calc width diff
    \addtolength\mylenA{-\the\wd\myboxB}%
    \ifdim\wd\myboxB<\wd\myboxA%
       \rlap{\hskip 0.5\mylenA\usebox\myboxB}{\usebox\myboxA}%
    \else
        \hskip -0.5\mylenA\rlap{\usebox\myboxA}{\hskip 0.5\mylenA\usebox\myboxB}%
    \fi}
\makeatother
\renewcommand{\arraystretch}{1.2}

\makeatletter
\renewcommand*\env@matrix[1][\arraystretch]{%
  \edef\arraystretch{#1}%
  \hskip -\arraycolsep
  \let\@ifnextchar\new@ifnextchar
  \array{*\c@MaxMatrixCols c}}
\makeatother

\usepackage{esint}
\usepackage{setspace}
\usepackage{colortbl}
\usepackage{xcolor}
\usepackage{enumitem}

\usepackage{float}
\usepackage{fancyhdr}
\usepackage{lastpage,refcount, atbegshi}
\usepackage[hidelinks]{hyperref}
\AtBeginShipout{%
  \ifnum\value{page}=\number\numexpr\getpagerefnumber{LastPage}-2\relax
    \phantomsection\label{preLastPage}
  \fi}

% \usepackage{polyglossia}

% \setmainfont{David CLM}
% \setsansfont{Miriam CLM}

% \setdefaultlanguage{english}
% \setotherlanguage{english}  

% \setmathrm{TeX Gyre Schola}

\makeatletter
\def\xpg@set@normalfont#1{%
  \letcs{\rmfamily}{#1@font@rm}%
  \letcs{\sffamily}{#1@font@sf}%
  \letcs{\ttfamily}{#1@font@tt}%
  \def\normalfont{\protect\xpg@select@fontfamily{#1}}%def instead of gdef
  \gdef\reset@font{\protect\normalfont}%
}
% \addto\inlineextras@english{\xpg@set@normalfont{english}}
% \addto\blockextras@english{\xpg@set@normalfont{english}}
\makeatother

\usepackage[style=english]{csquotes}

\usepackage[continuous]{pagenote}
\renewcommand*{\notesname}{NOTE NAME}
\renewcommand{\sectionname}{SECTION NAME}
\renewcommand{\notenuminnotes}[1]{{\textnormal#1. }}
\makepagenote

\renewcommand\text[1]{\textnormal{\textenglish{#1}}}

\setlength{\parindent}{0pt}
\pagestyle{fancy}     % Set the page style to fancy
\fancyhf{}            % Clear all header and footer fields
\renewcommand{\headrulewidth}{0pt} % No line in header
\setlength{\headsep}{0.2cm}% Set the space between header and text

% Define header content
\fancyhead[L]{Page: {\thepage}} % Left header
\fancyhead[R]{\ifthenelse{\value{page}=1}{\today}{} } % Right header shows date on first page

\def\imagewidth{0.9}

\newenvironment{cheatformula}[1][כותרת]{
    \begin{minipage}{\linewidth}
    \textbf{#1}:
}{
    % changing vertical  space between every cheat formula
    \end{minipage}\\[0ex]
}

\newcommand{\cheatimage}[4][\imagewidth]{
    \begin{figure}[H]
        \centering
        \includegraphics[width=#1\linewidth]{#2}
        \caption{#3}
        \label{#4}
    \end{figure}
}

\newcommand*{\NameAndID}{%
    \par\noindent\makebox[2in]{\hrulefill} \hspace{0.5in}\makebox[2.0in]{\hrulefill}%
    \par\noindent\makebox[2in][r]{Ramtin Behesht Aeen}      \hspace{0.5in}\makebox[2.0in][r]{ramtinba145822@gmail.com}%
}%

\title{Summary of }
\author{Auther}

\begin{document}

\setlength{\columnsep}{1cm}
\setlength{\columnseprule}{1pt}

\makeatletter
\begin{center}
    {\NameAndID}\\[2ex]
    {\Large{\textbf{\@title}}}\\[2ex]
\end{center}
\makeatother

\begin{multicols*}{2}
\raggedcolumns


\section{Neural Network}
\begin{cheatformula}[Neural Network Structure]

Simple Python Code:
\begin{lstlisting}
  class Layer:
    def __init__(self):
        self.inp = None
        self.out = None

    def __call__(self, inp: np.ndarray) -> np.ndarray:
        return self.forward(inp)

    def forward(self, inp: np.ndarray) -> np.ndarray:
        raise NotImplementedError

    def backward(self, up_grad: np.ndarray) -> np.ndarray:
        raise NotImplementedError

    def step(self, lr: float) -> None:
        pass
\end{lstlisting}
\begin{itemize}
  \item[$\circ$] Input features:
  \begin{equation}
a  
  \end{equation}

  \item[$\circ$] Features weights:
  \begin{equation}
  a
  \end{equation}

  \item[$\circ$] Bias term:
  \begin{equation}
  a
  \end{equation}

  
  \item[$\circ$] Activation function :
  \begin{equation}
  a
  \end{equation}

  
  \item[$\circ$] Output of the neuron:
  \begin{equation}
  y
  \end{equation}
\end{itemize}

\begin{lstlisting}
  def greet(name):
      print(f"Hello, {name}!")
  
  greet("World")
  \end{lstlisting}

\end{cheatformula}

\begin{cheatformula}[Squared Error(SE)] 
Most Common error function in linear regression is:
\begin{equation}
SE: ( y^{(i)} - h(x^{(i)}, w))^{2}
\end{equation}
\end{cheatformula}



\begin{cheatformula}[Sum of Squared Errors (SSE)]
Cost function should measure all predictions. Thus a choice could be Sum of Squared Error(SE)
\begin{equation}
SSE: \sum_{i=1}^{N}(y^{(i)} - h(x^{(i)}, w) )
\end{equation}
\end{cheatformula}

\begin{cheatformula}[Solve it analytically for one dimension]
Predicted:
\begin{equation}
\widehat{y} = w_{0} + w_{1}x
\end{equation}
SSE or Cost Function:
\begin{equation}
J(w_{0}, w_{1}) := \sum_{i=1}^{N}(y^{(i)} - \widehat{y}^{(i)})^{2} = \sum_{i=1}^{N}(y^{(i)} -  w_{0} + w_{1}x^{(i)})^{2}
\end{equation}
Assumptions:
\begin{equation}
\frac{\partial J}{\partial w_{0}} = 0 \,, \frac{\partial J}{\partial w_{1}} = 0
\end{equation}
$\frac{\partial J}{\partial w_{0}} = 0:$ thus:
\begin{equation}
\frac{\partial }{\partial w_{0}}  ( \sum_{i=1}^{N}(y^{(i)} -  (w_{0} + w_{1}x^{(i)}))^{2}) = 0  \, \footnotemark
\end{equation}

\begin{equation}
-2 \sum_{i=1}^{N}(y^{(i)} -  w_{0} - w_{1}x^{(i)}) = 0
\end{equation}
\footnotetext{\textsuperscript{\thefootnote} $fog(x)' = f(g(x))' g(x)'$}

For this equation to equal zero, the following condition must be met:
\begin{itemize}
\item[$\circ$] $\sum_{i=1}^{N}y^{(i)} = 0 := Y$ 
\item[$\circ$] $\sum_{i=1}^{N}-w_{0}  = 0 := n w_{0} $ 
\item[$\circ$] $\sum_{i=1}^{N}-w_{1}x^{(i)}  = 0 := X $ 
\end{itemize}
Thus:
\begin{equation}
0 = Y - n w_{0} -w_{1}x^{(i)} \longrightarrow  w_{0} = \frac{(Y -w_{1}x^{(i)})}{n}
\end{equation}
\end{cheatformula}


\pagebreak




\end{multicols*}
\newpage

\end{document}